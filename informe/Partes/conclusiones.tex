\par En el siguiente documento \footnote{http://www.net.in.tum.de/fileadmin/TUM/NET/NET-2012-08-1/NET-2012-08-1_02.pdf} Martin Jobst plantea varios casos de comportamientos anómalos en los traceroutes que uno puede llegar a obtener. En nuestros experimentos nos hemos encontrado con varios de ellos, por ejemplo en los traceroutes a Ciudad del cabo y a Kazajthan hubo algunos hops que no respondieron. Aún así este error no impactó en el análisis que realizamos ya que no perdimos ningún salto intercontinental en el medio.
\par También comenta que quien no responde nuestros mensajes puede ser el último nodo de nuestro camino, es decir el destino de nuestra ruta. Este problema lo hemos tenido en algunas universidades con las cuales habíamos experimentado, pero como la mayor parte de los hops de su traceroute tampoco habían respondido no teníamos material suficiente para analizar la ruta. En particular, las universidades que no nos respondieron están ubicadas en China y en Corea del Norte.
\par Por último, otro problema que tuvimos en los tres experimentos fue que el RTT que nos reportaba era falso, en particular porque el RTT hasta algunos hops era menor que el de su predecesor, Jobst comenta que una causa posible de esta anomalía es que las tablas de ruteo generan que el camino de regreso no sea el mismo que el de partida. El caso más problemático fue el traceroute a Jamaica, que creemos que desde su último hop se genera una ruta más directa hacia Argentina. \\
\par Por otro lado, por la naturaleza del método de Cimbala, se espera que la cantidad de saltos intercontinentales sea despreciable frente a la cantidad de saltos a hops cercanos. De esta manera si para llegar a destino hay que realizar varios saltos intercontinentales (como el caso del segundo experimento) y no hay un gran recorrido intracontinental para compensar no se van a poder detectar los saltos intercontinentales.
\par También pensamos la posibilidad de generar un valor de corte fijo para ZRTT y creemos que de esa manera no se puede subsanar la falla que posee ese método. Aún así las predicciones del método de Cimbala en casos normales predice con muy buena exactitud.
